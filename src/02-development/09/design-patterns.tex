\documentclass{a4beamer}
\input{lecture-common.def}

\def\classfield#1{\underbar{#1}}


\lecturetitle{Программная инженерия. Лекция №9 — ООП. Шаблоны проектирования.}
\title[ООП. Шаблоны проектирования]{Объектно-ориентированное проектирование. Шаблоны~проектирования}
\author{Алексей Островский}
\institute{\small{Физико-технический учебно-научный центр НАН Украины}\vspace{2ex}}
\date{21 ноября 2014 г.}

\begin{document}
	\frame{\titlepage}

	\section[ООП]{Объектно-ориентированное проектирование}

	\frame{
		\frametitle{Объектно-ориентированное проектирование}

		\begin{Definition}
			\textbf{Объектно-ориентированное проектирование} \engterm{object-oriented design} — 
			решение задачи проектирования программной системы с~использованием объектов и~взаимодействий между~ними.
		\end{Definition}

		\vspace{1ex}
		\begin{Definition}
			\textbf{Объект} — сильная связь между структурами данных и методами ($\simeq$ функциями), 
			обрабатывающими эти данные. 
		\end{Definition}

		\vspace{1ex}
		\textbf{Составляющие объекта:} 
		\begin{itemize}
			\item идентификатор; 
			\item свойства; 
			\item методы.
		\end{itemize}
	}

	\subsection{Концепции ООП}

	\frame{	
		\frametitle{Концепции ООП}

		\begin{itemize}
			\item \textbf{Объекты};

			\item
			\textbf{инкапсуляция} — скрытие информации от~внешних (по~отношению к~системе/объекту) сущностей;

			\item
			\textbf{наследование} — повторное использование методов работы с~данными в~различных условиях; 
			дополнение функциональности объектов;

			\item
			\textbf{интерфейсы} — формальное описание методов и~данных, используемое для~взаимодействия 
			между~объектами; разделение описания и~имплементации;

			\item
			\textbf{полиморфизм} — возможность использования наследованных от~объекта потомков в~том~же~контексте, что~и~сам~объект.
		\end{itemize}
	}

	\subsection{Процесс ООП}

	\frame{
		\frametitle{Процесс ООП}

		\textbf{Входные данные} для ООП:
		\begin{itemize}
			\item
			концептуальная модель (диаграмма классов UML с основными понятиями предметной области, независимая от реализации);
			\item
			варианты применения;
			\item
			диаграммы последовательности для вариантов применения;
			\item
			реляционная модель данных (может разрабатываться параллельно с объектами).
		\end{itemize}
	}

	\frame{
		\frametitle{Процесс ООП (продолжение)}

		\textbf{Процесс ООП:}
		\begin{itemize}
			\item 
			разграничение объектов, составление диаграммы классов (сущности $\Rightarrow$ объекты);
			\item
			конкретизация диаграмм последовательности;
			\item
			определение контекста: используемых библиотек, КПИ, …; 
			\item
			подбор и реализация шаблонов проектирования;
			\item
			определение взаимодействия объектов с источниками данных (базы данных, удаленные объекты).
		\end{itemize}
	}

	\section[Шаблоны]{Шаблоны проектирования}

	\frame{
		\frametitle{Шаблоны проектирования}

		\begin{Definition}
			\textbf{Шаблон проектирования} \engterm{design pattern} — типовой конструктивный элемент программной системы, 
			задающий взаимодействие нескольких компонентов системы, а~также~роли и~сферы ответственности исполнителей. 
		\end{Definition}

		\vspace{1ex}
		\begin{figure}
			{\small\begin{tikz*}[%
	every node/.style={rectangle,align=center,minimum height=3em},
	every label/.style={,minimum height=0pt,font=\small\itshape}
]
	\node(comp) {Проектирование \\ компонентов};
	\node(constr) [right=of comp] {Конструирование};
	\node(arch) [above=of comp] {Архитектура};
	\node(model) [above=of constr] {Моделирование};
	\node(req) [right=of model] {Инженерия \\ требований};
	\node(test) [right=of constr] {Тестирование};
	\node(maint) [right=of test] {Сопровождение};
	
	\draw[<->] (req) to (model);
	\draw[->] (constr) to (test);
	\draw[->] (test) to (maint);
	\draw[->] (model) to (arch);
	\draw[->] (arch) to (comp);
	\draw[->] (comp) -- node(_ref)[coordinate]{} (constr);
	
	\node(design) [draw,dashed,
		fit=(model.north east) (arch.north west) (comp.south west) (constr.south east),label=above:Проектирование и программирование] {};

	\node [note={(_ref)},below=3em of _ref,font=\small\bfseries] {%
		шаблоны применяются тут
	};
\end{tikz*}
}
			\caption{Роль шаблонов в разработке ПО}
		\end{figure}
	}

	\subsection{Составляющие шаблонов}

	\frame{
		\frametitle{Составляющие шаблонов}

		\textbf{Составляющие:}
		\begin{itemize}
			\item название;
			\item область применения, описание проблемы, которую решает шаблон проектирования;
			\item обобщенная структура шаблона: основные компоненты, их~взаимоотношения и~выполняемые функции 
			(на~естественном языке или~диаграмма классов UML);
			\item результат применения шаблона, возможные отрицательные последствия.
		\end{itemize}

		\vspace{1ex}
		\textbf{Чем не являются шаблоны:}
		\begin{itemize}
			\item
			\textbf{Шаблон $\neq$ архитектура:} архитектура системы более абстрактна, 
			шаблон подразумевает конкретную реализацию;

			\item
			\textbf{Шаблон $\neq$ КПИ:} шаблон требует имплементации, КПИ — это~готовый~код.
		\end{itemize}
	}

	\subsection{Классификация шаблонов}

	\frame{
		\frametitle{Классификация шаблонов проектирования}

		\textbf{Область применения:} 
		\begin{itemize}
			\item общего назначения; 
			\item для конкретной предметной области (пользовательский интерфейс, защита информации, веб-дизнайн, …)
		\end{itemize}

		\vspace{1ex}
		\textbf{Уровень проектирования:}
		\begin{itemize} 
			\item уровень архитектуры; 
			\item уровень отдельных компонентов.
		\end{itemize}

		\vspace{1ex}
		\textbf{Цель применения:} 
		\begin{itemize}
			\item порождающие шаблоны \engterm{creational patterns};
			\item структурные шаблоны \engterm{structural patterns};
			\item поведенческие шаблоны \engterm{behavioral patterns}.
		\end{itemize}
	}

	\section[Примеры]{Примеры шаблонов}

	\frame{
		\frametitle{Примеры шаблонов}

		\textbf{Порождающие:} фабрика, \inlink{sec:Builder}{строитель}, \inlink{sec:Singleton}{одиночка}, прототип, …

		\vspace{1ex}
		\textbf{Структурные:} адаптер, \inlink{sec:Bridge}{мост}, \inlink{sec:Decorator}{декоратор}, фасад, компоновщик, Proxy, …

		\vspace{1ex}
		\textbf{Поведенческие:} \inlink{sec:Iterator}{итератор}, \inlink{sec:Observer}{наблюдатель}, команда, 
			состояние, хранитель, посредник, цепочка обязанностей, …

		\vspace{2ex}
		\textbf{Архитектурные:} \inlink{sec:AR}{ActiveRecord}, Data Mapper, ленивая загрузка, …

		\vspace{1ex}
		\textbf{Параллелизация:} блокировка, семафоры, монитор, пул нитей исполнения, …
	}

	\subsection{ActiveRecord}\label{sec:AR}

	\frame{
		\frametitle{Архитектурный шаблон: ActiveRecord}

		\begin{figure}
			\begin{tikz*}[%
	class/.style={draw,rectangle split,rectangle split parts=3,align=left}
]
	\node(AR) [class,text width=8em] {%
		\hfill\textbf{ActiveRecord}\hfill\strut{}
		\nodepart{two}
		$+$ Поле 1 \\
		$+$ Поле 2 \\
		$+$ Поле $n$
		\nodepart{three}
		$+$ \classfield{создать()} \\
		$+$ \classfield{поиск()} \\
		$+$ сохранить() \\
		$+$ удалить()
	};

	\node(table) [text width=20em,right=4em of AR,align=center,draw] {%
		\textbf{База данных} \\[.5ex]
		\renewcommand{\arraystretch}{1.5}%
		\begin{tabular}{|c|c|c|c|}
			\hline
			\textbf{Поле 1} & \textbf{Поле 2} & … & \textbf{Поле $n$} \cr
			\hline
			$a_{11}$ & $a_{12}$ & … & $a_{1n}$ \cr
			\hline
			$a_{21}$ & $a_{22}$ & … & $a_{2n}$ \cr
			\hline
			\multicolumn{4}{c}{$\vdots$}
		\end{tabular}
	};

	\draw[<->] (AR) -- (table);
\end{tikz*}

			\caption{Шаблон доступа к БД ActiveRecord. Черта снизу в UML обозначает статические поля и методы. 
				«+» обозначает общедоступные поля/методы.}
		\end{figure}
	}

	\frame{
		\frametitle{ActiveRecord — описание}

		\begin{tabular}{lp{0.75\textwidth}}
			\textbf{Название:} & ActiveRecord \cr
			\textbf{Проблема:} & \raggedright обеспечение доступа к~реляционным базам данных 
				в~объектно-ориентированных приложениях. \cr
			\textbf{Решение:} & \raggedright Каждой таблице (представлению) в БД соответствует свой класс; 
				каждой строке таблицы — экземпляр класса; столбцам таблицы — поля объекта. 
				В классе определены методы для~сохранения объекта в~БД, удаления и~поиска. 
				Ключи (foreign key) определяют отношения между классами AR. \cr
			\textbf{Недостатки:} & \raggedright избыточное количество и/или непрозрачность запросов к~СУБД (ср.~с~DataMapper); 
				проблема идентичности структуры класса и~таблицы БД. \cr
			\textbf{Примеры:} & \raggedright в составе MVC в~веб-фреймворках (напр., CakePHP, Propel, Yii, Ruby on Rails).
		\end{tabular}
	}

	\subsection{Singleton}\label{sec:Singleton}

	\frame{
		\frametitle{Порождающий шаблон: Singleton}

		\begin{figure}
			\begin{tikz*}[%
	class/.style={draw,rectangle split,rectangle split parts=3,align=left}
]
	\node [class,text width=15em] {
		\hfill\textbf{Singleton}\hfill\strut{}
		\nodepart{two}
		$-$ \classfield{instance} : Singleton
		\nodepart{three}
		$+$ \classfield{getInstance()} : Singleton \\
		$-$ Singleton() : void
	};
\end{tikz*}

			\caption{UML-диаграмма классов для шаблона Singleton}
		\end{figure}
	}

	\frame{
		\frametitle{Singleton — описание}

		\begin{tabular}{lp{0.75\textwidth}}
			\textbf{Название:} & Singleton (одиночка) \cr
			\textbf{Проблема:} & \raggedright необходимость в строго одном объекте определенного класса 
				(напр., для координации действий в системе; из соображений производительности). \cr
			\textbf{Решение:} & \raggedright публичный статический метод для доступа к объекту, 
				создающий при~необходимости экземпляр класса и~сохраняющий~его в~скрытой статической переменной. \cr
			\textbf{Недостатки:} & \raggedright усложнение тестирования; введение скрытых зависимостей 
				(\extlink{https://code.google.com/p/google-singleton-detector/wiki/WhySingletonsAreControversial}{детальнее}). \cr
			\textbf{Примеры:} & Системы ведения логов. \cr
			\textbf{Замена:} & \extlink{https://en.wikipedia.org/wiki/Dependency_injection}{внедрение зависимости}.
		\end{tabular}
	}

	\frame{
		\frametitle{Singleton — реализация}

		\lstinputlisting[language=java]{code-Singleton.java}
	}

	\subsection{Builder}\label{sec:Builder}

	\frame{
		\frametitle{Порождающий шаблон: Builder}

		\begin{figure}
			\begin{tikz*}[%
	class/.style={draw,rectangle split,rectangle split parts=3,align=left}
]
	\node(builder) [class,text width=12em] {%
		\hfill\textbf{\textit{Builder}}\hfill\strut{}
		\nodepart{two}
		\nodepart{three}
		$+$ \textit{buildPart1()} : void \\
		$+$ \textit{buildPart2()} : void \\
		\hfill $\vdots$ \hfill\strut{} \\
		$+$ \textit{create()} : Product
	};
	
	\node(cbuilder) [class,text width=12em,right=6em of builder] {%
		\hfill\textbf{ConcreteBuilder}\hfill\strut{}
		\nodepart{two}
		\nodepart{three}
		$+$ buildPart1() : void \\
		$+$ buildPart2() : void \\
		\hfill $\vdots$ \hfill\strut{} \\
		$+$ create() : Product
	};

	\node(product) [class,text width=12em,below=4em of cbuilder]{%
		\hfill\textbf{Product}\hfill\strut{}
		\nodepart{two}
		\nodepart{three}
	};

	\draw[->,>=open triangle 60] (cbuilder) to (builder);
	\draw[->,dashed] (cbuilder) -- node[left] {«создает»} (product);
\end{tikz*}

			\caption{UML-диаграмма классов для шаблона Builder. \textit{Курсивом} в UML обозначаются интерфейсы и абстрактные методы.}
		\end{figure}
	}

	\frame{
		\frametitle{Builder — описание}

		\begin{tabular}{lp{0.75\textwidth}}
			\textbf{Название:} & Builder (строитель) \cr
			\textbf{Проблема:} & \raggedright создание объектов с заданным набором свойств 
				без~имплементации большого количества конструкторов. \cr
			\textbf{Решение:} & \raggedright использование служебного объекта для пошагового задания 
				свойств и~создания результирующего объекта. \cr
			\textbf{Преимущества:} & \raggedright тип объекта может варьироваться в засимости от заданных параметров. \cr
			\textbf{Примеры:} & \raggedright создание документов с жесткой структурой (SQL-запросов, XML/HTML-документов).
		\end{tabular}
	}

	\frame{
		\frametitle{Builder — реализация}

		\lstinputlisting[language=java]{code-Builder.java}
	}

	\subsection{Bridge}\label{sec:Bridge}

	\frame{
		\frametitle{Структурный шаблон: Bridge}

		\begin{figure}
			\begin{tikz*}[%
	class/.style={draw,rectangle split,rectangle split parts=3,align=left},
	code/.style={draw,rectangle,minimum height=2.5em,font=\ttfamily\small}
]
	\node(abs) [class,text width=12em] {%
		\hfill\textbf{Abstraction}\hfill\strut{}
		\nodepart{two}
		$-$ engine : Engine
		\nodepart{three}
		$+$ function()
	};
	\node(abs-function) [code,below=1em of abs.south east,anchor=north west] {
		\textbf{this}.engine.run();
	};
	\draw[o-,dashed] ($ (abs.south east) + (-2em,1em) $) |- (abs-function);

	\node(r-abs) [class,text width=12em,below=6em of abs] {%
		\hfill\textbf{RefinedAbstraction}\hfill\strut{}
		\nodepart{two}
		\nodepart{three}
		$+$ refinedFunction()
	};
	\node(r-abs-function) [code,below=1em of r-abs.south east,anchor=north west] {
		\textbf{this}.function();
	};
	\draw[o-,dashed] ($ (r-abs.south east) + (-2em,1em) $) |- (r-abs-function);

	\node(engine) [class,text width=12em,right=8em of abs] {%
		\hfill\textbf{\textit{Engine}}\hfill\strut{}
		\nodepart{two}
		\nodepart{three}
		$+$ run()
	};
	\node(cengine) [class,text width=12em,below=6em of engine] {%
		\hfill\textbf{ConcreteEngine}\hfill\strut{}
		\nodepart{two}
		\nodepart{three}
		$+$ run()
	};

	\draw[->,>=open diamond] (engine) to (abs);
	\draw[->,>=open triangle 60] (r-abs) to (abs);
	\draw[->,>=open triangle 60] (cengine) to (engine);
\end{tikz*}

			\caption{UML-диаграмма классов для шаблона Bridge}
		\end{figure}
	}

	\frame{
		\frametitle{Bridge — описание}

		\begin{tabular}{lp{0.75\textwidth}}
			\textbf{Название:} & Bridge (мост) \cr
			\textbf{Проблема:} & \raggedright отделение функциональности, предоставляемой интерфейсом, 
				от~конкретной имплементации, чтобы они могли меняться независимо. \cr
			\textbf{Решение:} & \raggedright выделение методов с несколькими реализациями 
				в~отдельные классы; создание интерфейса, общего для~всех~реализаций. \cr
			\textbf{Недостатки:} & \raggedright возможно излишнее усложнение кода при наличии одной имплементации. \cr
			\textbf{Примеры:} & \raggedright мультиплатформенные графические интерфейсы (Java AWT, Qt).
		\end{tabular}
	}

	\frame{
		\frametitle{Bridge — реализация}

		\lstinputlisting[language=java]{code-Bridge.java}
	}

	\subsection{Decorator}\label{sec:Decorator}

	\frame{
		\frametitle{Структурный шаблон: Decorator}

		\begin{figure}
			\begin{tikz*}[%
	class/.style={draw,rectangle split,rectangle split parts=3,align=left},
	code/.style={draw,rectangle,minimum height=2.5em,font=\ttfamily\small}
]
	\node(component) [class,text width=12em] {%
		\hfill\textbf{\textit{Component}}\hfill\strut{}
		\nodepart{two}
		\nodepart{three}
		$+$ \textit{operation()}
	};

	\node(ccomponent) [class,below left=4em and -0.5em of component,text width=12em] {%
		\hfill\textbf{ConcreteComponent}\hfill\strut{}
		\nodepart{two}
		\nodepart{three}
		$+$ operation()
	};

	\node(decorator) [class,below right=4em and -0.5em of component,text width=12em] {%
		\hfill\textbf{Decorator}\hfill\strut{}
		\nodepart{two}
		$-$ component
		\nodepart{three}
		$+$ operation()
	};
	\node(deco-operation) [code,below=1em of decorator.south,anchor=north east] {%
		\textbf{this}.component.operation();
	};
	\draw[o-,dashed] ($ (decorator.three east) + (-1em,0) $) |- (deco-operation);

	\node(_tmp) [coordinate,below=2em of component.south]{};
	\draw (ccomponent.north) |- (_tmp);
	\draw (decorator.north) |- (_tmp);
	\draw[->,>=open triangle 60] (_tmp) -- (component.south);
\end{tikz*}

			\caption{UML-диаграмма классов для шаблона Decorator}
		\end{figure}
	}

	\frame{
		\frametitle{Decorator — описание}

		\begin{tabular}{lp{0.75\textwidth}}
			\textbf{Название:} & Decorator (декоратор) \cr
			\textbf{Проблема:} & \raggedright Изменение поведения конкретного объекта (при~сохранении интерфейса), 
				а~не~его~класса в~целом. \cr
			\textbf{Решение:} & \raggedright Создание класса с интерфейсом исходного класса,
				который направляет вызовы методов исходному объекту после~определенной обработки. \cr
			\textbf{Недостатки:} & \raggedright усложнение читаемости кода; 
				некорректное использование вместо создания подклассов. \cr
			\textbf{Примеры:} & \raggedright ввод/вывод в Java; по~аналогичному принципу работают декораторы в~Python.
		\end{tabular}
	}

	\frame{
		\frametitle{Decorator — реализация}

		\lstinputlisting[language=java]{code-Decorator.java}
	}

	\subsection{Iterator}\label{sec:Iterator}

	\frame{
		\frametitle{Поведенческий шаблон: Iterator}

		\begin{figure}
			\begin{tikz*}[%
	class/.style={draw,rectangle split,rectangle split parts=3,align=left},
	qty/.style={font=\footnotesize}
]
	\node(container) [class,text width=12em] {%
		\hfill\textbf{Container}\hfill\strut{}
		\nodepart{two}
		$-$ elements
		\nodepart{three}
		$+$ iterator() : Iterator
	};

	\node(element) [class,right=8em of container,text width=12em] {%
		\hfill\textbf{Element}\hfill\strut{}
		\nodepart{two}
		\nodepart{three}
	};

	\node(iterator) [class,below=5em of element,text width=12em] {%
		\hfill\textbf{\textit{Iterator}}\hfill\strut{}
		\nodepart{two}
		\nodepart{three}
		$+$ \textit{hasNext()} : bool \\
		$+$ \textit{next()} : Element
	};

	\node(citerator) [class,text width=12em] at (container.south |- iterator.west) {%
		\hfill\textbf{ConcreteIterator}\hfill\strut{}
		\nodepart{two}
		\nodepart{three}
		$+$ hasNext() : bool \\
		$+$ next() : Element
	};

	\draw[->,>=open diamond] (element) -- (container)
		node[qty,very near start,below]{0..*}
		node[qty,very near end,below]{0..*};
	\draw[->,>=open triangle 60] (citerator) to (iterator);
	\draw[->,dashed] (container) -- node[right]{«создает»} (citerator);
	\draw[->,dashed] (iterator) -- node[left]{«перечисляет»} (element);
\end{tikz*}

			\caption{UML-диаграмма классов для шаблона Iterator}
		\end{figure}
	}

	\frame{
		\frametitle{Iterator — описание}

		\begin{tabular}{lp{0.75\textwidth}}
			\textbf{Название:} & Iterator (итератор) \cr
			\textbf{Проблема:} & \raggedright Отделение функциональности последовательного доступа 
				к~элементам контейнера от~внутренней структуры контейнера. \cr
			\textbf{Решение:} & \raggedright Создание объекта-итератора, возвращаемого контейнером 
				и~содержащего в~себе необходимую функциональность. \cr
			\textbf{Недостатки:} & \raggedright некоторые алгоритмы не~могут использовать итераторы, 
				так~как они~зависят от~внутренней структуры контейнера. \cr
			\textbf{Примеры:} & \raggedright коллекции в большинстве языков программирования.
		\end{tabular}
	}

	\frame{
		\frametitle{Iterator — реализация}

		\lstinputlisting[language=java]{code-Iterator.java}
	}

	\subsection{Observer}\label{sec:Observer}

	\frame{
		\frametitle{Поведенческий шаблон: Observer}

		\begin{figure}
			\begin{tikz*}[%
	class/.style={draw,rectangle split,rectangle split parts=3,align=left},
	code/.style={draw,rectangle,minimum height=2.5em,font=\ttfamily\small}
]
	\node(subject) [class,text width=12em] {%
		\hfill\textbf{Subject}\hfill\strut{}
		\nodepart{two}
		$-$ observers: Observer[]
		\nodepart{three}
		$+$ register(observer) \\
		$+$ unregister(observer) \\
		$-$ notifyObservers()
	};
	\node(subject-code) [code,below=1em of subject.south,align=left]  {%
		notifyObservers: \\
		\hspace{2em}\textbf{for} obs \textbf{in} \textbf{this}.observers: \\
		\hspace{4em}obs.notify();
	};
	\draw[dashed] (subject) -- (subject-code);

	\node(observer) [class,left=8em of subject,text width=12em] {%
		\hfill\textbf{\textit{Observer}}\hfill\strut{}
		\nodepart{two}
		\nodepart{three}
		$+$ \textit{notify()}
	};

	\node(c-observer) [class,below=4em of observer,text width=12em] {%
		\hfill\textbf{ConcreteObserver}\hfill\strut{}
		\nodepart{two}
		\nodepart{three}
		$+$ notify()
	};

	\draw[->,>=open diamond] (observer) to (subject);
	\draw[->,>=open triangle 60] (c-observer) to (observer);
\end{tikz*}

			\caption{UML-диаграмма классов для шаблона Observer}
		\end{figure}
	}

	\frame{
		\frametitle{Observer — описание}

		\begin{tabular}{lp{0.75\textwidth}}
			\textbf{Название:} & Observer (наблюдатель) \cr
			\textbf{Проблема:} & \raggedright своевременное обновление состояния 
				для~зависимых друг~от~друга объектов. \cr
			\textbf{Решение:} & \raggedright Хранение списка зависимых объектов 
				и~уведомление~их об~изменении состояния. \cr
			\textbf{Недостатки:} & \raggedright Утечки памяти (зависимые объекты хранятся в~памяти 
				до~явного удаления зависимости с~помощью метода \code{unregister}). \cr
			\textbf{Примеры:} & \raggedright системы графического пользовательского интерфейса.
		\end{tabular}
	}

	\frame{
		\frametitle{Observer — реализация}

		\lstinputlisting[language=java]{code-Observer.java}
	}

	\section{Заключение}

	\subsection{Выводы}
	
	\frame{
		\frametitle{Выводы}
		
		\begin{enumerate}
			\item
			Объектно-ориентированное проектирование — один из~основных подходов к~проектированию ПО. 
			В~его~рамках предметная область разбивается на объекты, взаимодействующие между собой.

			\vspace{0.5ex}
			\item
			Ключевые понятия ООП — наследование, полиморфизм, инкапсуляция, интерфейсы.

			\vspace{0.5ex}
			\item
			В рамках ООП часто используются стандартные элементы (шаблоны) программных систем. 
			Выделяют архитектурные, порождающие, структурные и~поведенческие шаблоны.
		\end{enumerate}
	}
	
	\subsection{Материалы}
	
	\frame{
		\frametitle{Материалы}
		
		\begin{thebibliography}{9}
			\bibitem[1]{1}
			Gamma, Erich et al.
			\newblock Design Patterns.
			\newblock {\footnotesize Addison-Wesley, 1995.}
		
			\bibitem[2]{2}
			Fowler, Martin et al.
			\newblock Patterns of Enterprise Application Architecture.
			\newblock {\footnotesize Addison-Wesley, 2002.}

			\bibitem[3]{3}
			Ward Cunningham et al.
			\newblock Portland Pattern Repository.
			\newblock {\footnotesize \url{http://c2.com/cgi/wiki?DesignPatterns}}
			\newblock {\footnotesize (Вики по шаблонам проектирования. По совместительству — первая вики в мире.)}
		\end{thebibliography}
	}
	
	\frame{
		\frametitle{}
		
		\begin{center}
			\Huge Спасибо за внимание!
		\end{center}
	}
\end{document}
