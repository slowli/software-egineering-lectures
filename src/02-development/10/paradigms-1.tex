\documentclass{a4beamer}
\input{lecture-common.def}


\lecturetitle{Программная инженерия. Лекция №10 — Парадигмы программирования (часть 1).}
\title[Парадигмы программирования 1]{Парадигмы программирования (часть 1)}
\author{Алексей Островский}
\institute{\small{Физико-технический учебно-научный центр НАН Украины}\vspace{2ex}}
\date{28 ноября 2014 г.}

\begin{document}
	\frame{\titlepage}

	\section[Определение]{Определение парадигмы}

	\frame{
		\frametitle{Определение парадигмы}

		\begin{Definition}
			\textbf{Парадигма программирования} \engterm{programming paradigm} — совокупность идей и~понятий, 
			которые определяют общий стиль написания компьютерных программ, построения их~структуры и~отдельных элементов программной системы. 
		\end{Definition}

		\vspace{1ex}
		\textbf{Цель парадигмы программирования:}
		\begin{itemize}
			\item разделение программы на базовые составные элементы (напр., функции или~объекты);
			\item определение модели преобразования данных;
			\item внедрение ограничений на используемые конструкции.
		\end{itemize}
	}

	\subsection{Место парадигмы в разработке}

	\frame{
		\frametitle{Место парадигмы в разработке}

		\begin{figure}
			{\small\begin{tikz*}[%
	every node/.style={rectangle,align=center,minimum height=3em},
	every label/.style={,minimum height=0pt,font=\small\itshape}
]
	\node(comp) {Проектирование \\ компонентов};
	\node(constr) [right=of comp] {Конструирование};
	\node(paradigm) [above=of constr] {Парадигма пр.};
	\node(model) [above=of paradigm] {Моделирование};
	\node(arch) at (model.west -| comp.north) {Архитектура};
	\node(req) [right=of model] {Инженерия \\ требований};
	\node(test) [right=of constr] {Тестирование};
	\node(maint) [right=of test] {Сопровождение};
	
	\draw[<->] (req) to (model);
	\draw[->] (constr) to (test);
	\draw[->] (test) to (maint);
	\draw[->] (model) to (arch);
	\draw[->] (arch) to (comp);
	\draw[->] (comp) -- node(_ref)[near end,coordinate]{} (constr);
	
	\draw[->,dashed] (arch) to (paradigm);
	\draw[->,dashed] (paradigm) to (comp);
	\draw[->,dashed] (paradigm) to (constr);

	\node(design) [draw,dashed,
		fit=(model.north east) (arch.north west) (comp.south west) (constr.south east),label=above:Проектирование и программирование] {};
\end{tikz*}
}
			\caption{Роль парадигм программирования в~разработке~ПО. Парадигма выбирается, 
				исходя~из архитектуры системы и~влияет на~построение и~реализацию ее~компонентов}
		\end{figure}
	}

	\frame{
		\frametitle{Место парадигмы в разработке}

		\begin{figure}
			\begin{tikz*}[%
	block/.style={rectangle,draw,align=center,minimum height=3em,minimum width=10em},
	label/.style={font=\footnotesize}
]
	\node(se-process) [block] {Процесс ПИ};
	\node(lc-model) [block,right=12em of se-process] {Модель ЖЦ};
	\node(pl) [block,below=6em of se-process] {Язык \\ программирования};
	\node(paradigm) at (pl.east -| lc-model.south) [block] {Парадигма \\ программирования};

	\draw[->] (se-process) -- node[label,above] {характеризуется} (lc-model);
	\draw[->] (se-process) -- 
		node[label,midway,right] {использует} 
		node[label,very near end,left] {1..*} (pl);
	\draw[->] (pl) -- 
		node[label,very near start,below] {1..*}
		node[label,midway,above] {характеризуется} 
		node[label,very near end,below] {1..*} (paradigm);
\end{tikz*}

			\caption{Связь парадигм с языками программирования подобна связи процесса программной инженерии 
				с~выбранной моделью разработки}
		\end{figure}
	}

	\subsection{Классификация парадигм}

	\frame{
		\frametitle{Классификация парадигм программирования}

		{\small\begin{tikz*}[%
	every node/.style={rectangle,minimum height=2.25em,minimum width=7.5em}
]
	\node(paradigm) {\large\textbf{Программирование}};
	\node(declarative) [above=of paradigm] {Декларативное};
	\node(functional) [left=of declarative] {Функциональное};
	\node(logic) [right=of declarative] {Логическое};
	\node(imperative) [below=of paradigm] {Императивное};
	\node(struct) [below left=of imperative] {Структурное};
	\node(modular) [below=of imperative] {Модульное};
	\node(oo) [below right=of imperative] {ООП};

	\node(component) [below=of oo] {\textbf{Компонентное}};
	\node(service) [below=of component] {\textbf{Сервисное}};
	\node(aspect) [left=of service] {\textbf{Аспектное}};
	\node(agent) [right=of service] {\textbf{Агентное}};

	\draw[->] (paradigm) to (declarative);
	\draw[->] (paradigm) to (imperative);
	\draw[->] (declarative) to (functional);
	\draw[->] (declarative) to (logic);
	\draw[->] (imperative) to (struct);
	\draw[->] (imperative) to (modular);
	\draw[->] (imperative) to (oo);
	\draw[->] (oo) to (component);
	\draw[->] (component) to (service);
	\draw[->] (component) to (aspect);
	\draw[->] (component) to (agent);
\end{tikz*}
}
	}

	\frame{
		\frametitle{Декларативное программирование}

		\begin{Definition}	
		\textbf{Декларативное программирование} \engterm{declarative programming} — парадигма, 
		согласно которой программа представляет логику вычислений без~описания 
		прямой последовательности действий (действия определяются компилятором или~интерпретатором).
		\end{Definition}

		\vspace{1ex}
		\textbf{Области использования:} 
		\begin{itemize}
			\item математическое моделирование; 
			\item искусственный интеллект; 
			\item анализ данных;
			\item наука.
		\end{itemize}
	}

	\frame{
		\frametitle{Выполнение декларативной программы}

		\begin{figure}
			\begin{tikz*}[%
	every node/.style={rectangle,draw,minimum height=3em,minimum width=7em},
	label/.style={draw=none,minimum height=0pt,minimum width=0pt,font=\bfseries},
	seq/.style={rectangle split,rectangle split parts=4,font=\small},
	wrapper/.style={}
]
	\node(inter-seq) [seq] {
		Инструкция 1
		\nodepart{two}
		Инструкция 2
		\nodepart{three}
		…
		\nodepart{four}
		Инструкция $m$
	};
	\node(inter-lbl) [label,above=0.25em of inter-seq] {Интерпретатор};
	\node(inter) [wrapper,fit=(inter-lbl.north west) (inter-lbl.north east) (inter-seq.south west) (inter-seq.south east)] {};

	\node(program-seq) [seq,left=4em of inter-seq] {%
		Утверждение 1
		\nodepart{two}
		Утверждение 2
		\nodepart{three}
		…
		\nodepart{four}
		Утверждение $n$
	};
	\node(program-lbl) [label,above=0.25em of program-seq] {Программа};
	\node(program) [wrapper,fit=(program-lbl.north west) (program-lbl.north east) (program-seq.south west) (program-seq.south east)] {};

	\node(data) [above=of inter] {Данные};
	\node(memory) [below=of inter] {Память};
	
	\node(result) [right=of inter] {Результат};

	\draw[<->] (program) to (inter);
	\draw[->] (data) to (inter);
	\draw[<->] (memory) to (inter);
	\draw[->] (inter) to (result);
\end{tikz*}

			\caption{Декларативная программа не взаимодействует напрямую с памятью, поручая эту работу интерпретатору}
		\end{figure}
	}

	\frame{
		\frametitle{Императивное программирование}

		\begin{Definition}
			\textbf{Императивное программирование} \engterm{imperative programming} — парадигма, 
			согласно которой программа представляет собой последовательность действий, 
			изменяющих \emph{состояние программы}.
		\end{Definition}

		\vspace{1ex}
		\begin{Definition}
			\textbf{Состояние программы} \engterm{program state} — совокупность данных, 
			связанных со~всеми используемыми программой переменными в~конкретный момент времени.
		\end{Definition}

		\vspace{1ex}
		\textbf{Область использования:} 
		\begin{itemize}
			\item системные программы;
			\item прикладные программы.
		\end{itemize}
	}

	\frame{
		\frametitle{Выполнение императивной программы}

		\begin{figure}
			\begin{tikz*}[%
	every node/.style={rectangle,draw,minimum height=3em,minimum width=7em},
	label/.style={draw=none,minimum height=0pt,minimum width=0pt,font=\bfseries},
	seq/.style={rectangle split,rectangle split parts=4,font=\small},
	wrapper/.style={}
]
	\node(program-seq) [seq,left=4em of inter-seq] {%
		Инструкция 1
		\nodepart{two}
		Инструкция 2
		\nodepart{three}
		…
		\nodepart{four}
		Инструкция $m$
	};
	\node(memory) [right=of program-seq] {Память};

	\node(_center) [coordinate] at ($ (memory.center)!0.5!(program-seq.center) $) {};
	\node(_north) [coordinate] at (_center |- program-seq.north) {};
	\node(program-lbl) [label,above=0.25em of _north] {Программа};
	\node(program) [wrapper,fit=(program-lbl.north west) (program-lbl.north east) (program-seq.south west) (memory.south east)] {};

	\node(data) [left=of program] {Данные};
	
	\node(result) [right=of program] {Результат};

	\draw[<->] (program-seq) to (memory);
	\draw[->] (data) to (program);
	\draw[->] (program) to (result);
\end{tikz*}

			\caption{Императивная программа использует именованные области памяти (переменные) для хранения состояния вычислений}
		\end{figure}
	}

	\section[Декларативное пр.]{Декларативное программирование}

	\subsection{Функциональное программирование}

	\frame{
		\frametitle{Функциональное программирование}

		\begin{Definition}
		\textbf{Функциональное программирование} \engterm{functional programming} — парадигма, 
		согласно которой процесс исполнения программы представляется 
		последовательностью вычислений значений для математических функций.
		\end{Definition}

		\vspace{1ex}
		\textbf{Особенности:} 
		\begin{itemize}
			\item отказ от явного хранения переменных (функции без побочных эффектов);
			\item $\Rightarrow$ встроенная поддержка параллелизации, оптимизации и~кэширования 
			без~необходимости действий со~стороны программиста.
		\end{itemize}
	}

	\frame{
		\frametitle{Функциональное программирование (продолжение)}

		\begin{tikz*}[%
	every node/.style={rectangle,draw,align=center,minimum height=3em},
	function/.style={font=\bfseries}
]
	\node(input) [ellipse] {Входные \\ данные};
	\node(func1) [function,right=of input] {Функция 1};
	\node(func2) [function,right=of func1] {Функция 2};
	\node(func3) [function,below=of func2] {Функция 3};
	\node(output) [ellipse,right=of func2] {Результат};

	\draw[->] (input) to (func1);
	\draw[->] (func1) to (func2);
	\draw[->] (func1) |- (func3);
	\draw[->] (func3) to (func2);
	\draw[->] (func2.north) -- ++(0,1em) -| (func1.north);
	\draw[->] (func3.east) -| ($ (func3.south east) + (1em,-2em) $) -- ($ (func3.south) + (0em,-2em) $) -- (func3.south);
	\draw[->,dotted] (func2) to (output);
\end{tikz*}


		\vspace{1.5ex}
		Функции возвращают результат, иначе не меняя состояние программы (напр., через~переменные).

		\vspace{0.5ex}
		\textbf{Ключевые слова:} \extlink{http://en.wikipedia.org/wiki/Pure_function}{чистая функция}, 
			\extlink{http://en.wikipedia.org/wiki/Referential_transparency_\%28computer_science\%29}{прозрачность ссылок}.
	}

	\frame{
		\frametitle{Функциональное программирование (продолжение)}

		\textbf{Концепции:}
		\begin{itemize}
			\item
			функции высших порядков — функции, которые возвращают другие функции или~принимают функции в~качестве аргументов;
			\item
			замыкание \engterm{closure} —сохранение контекста функции при ее создании;
			\item
			рекурсия для создания циклов;
			\item
			ленивые вычисления \engterm{lazy evaluation} — вычисление аргументов функций по~мере~необходимости (не~при~задании).
		\end{itemize}

		\vspace{1ex}
		\textbf{Языки программирования:} Lisp, Scheme, Clojure, Erlang, Haskell, F\#.
	}

	\subsection{Логическое программирование}

	\frame{
		\frametitle{Логическое программирование}

		\begin{Definition}
			\textbf{Логическое программирование} \engterm{logic programming} — парадигма, согласно которой 
			программа представляет собой вывод с помощью правил формальной логики.
		\end{Definition}

		\vspace{1ex}
		\textbf{Правило вывода:} 
		\begin{center}
			$H \mathop{{:}{-}} B_1, \dots, B_n$, \quad т.\,е.\quad $H$, \textbf{если} $B_1$ \textbf{и} … \textbf{и} $B_n$.
		\end{center}

		\vspace{1ex}
		\textbf{Особенности:} 
		\begin{itemize}
			\item
			данные представляются как факты (безусловные утверждения, аксиомы); 
			\item
			результат выполнения — данные, для которых истинно некоторое утверждение.
		\end{itemize}
	}

	\frame{
		\frametitle{Логическое программирование (продолжение)}

		\textbf{Концепции:}
		\begin{itemize}
			\item
			процедурная интерпретация правил:
			\begin{center}
				$H \mathop{{:}{-}} B_1, \dots, B_n$;

				чтобы вычислить $H$, нужно вычислить $B_1$, …, $B_n$.
			\end{center}

			\vspace{0.75ex}
			\item
			контроль над стратегией доказательства утверждений 
			(параллельный или~последовательный поиск, перебор с~возвратом, …).
		\end{itemize}

		\vspace{1ex}
		\textbf{Языки программирования:} Prolog и диалекты, Oz.
	}

	\section[Императивное пр.]{Императивное программирование}

	\subsection{Структурное программирование}

	\frame{
		\frametitle{Структурное программирование}

		\begin{Definition}
			\textbf{Структурное программирование} \engterm{structured programming} — парадигма, 
			в основе которой лежит представление программы  в виде иерархии \emph{блоков}.
		\end{Definition}

		\vspace{1ex}
		\begin{Definition}
			\textbf{Блок инструкций} — логически связанный набор последовательных инструкций, предназначенный для: 
			\begin{itemize}
				\item ограничения области видимости; 
				\item обращения к блоку как к одной инструкции.
			\end{itemize}
		\end{Definition}

		\vspace{1ex}
		\textbf{Особенности:} 
		\begin{itemize}
			\item каждый блок должен иметь строго один вход и выход; 
			\item $\Rightarrow$ отказ от безусловных переходов (\code{\textbf{goto}}).
		\end{itemize}
	}

	\frame{
		\frametitle{Структурное программирование (продолжение)}

		\textbf{Элементы программ:}
		\begin{center}\linespread{1.0}
			\begin{tabular}{|l|p{0.4\textwidth}|}
				\hline
				\centering\textbf{Конструкция} & \centering\textbf{Псевдокод} \cr
				\hline
				последовательность &$ S_1; S_2; S_3; \dots $ \cr
				\hline
				ветвление & $\mathrm{if}\ C\ \mathrm{then}\ S_1\ \mathrm{else}\ S_2$ \cr
				\hline
				цикл & $\mathrm{while}\ C\ \mathrm{do}\ S$ \cr
				\hline
				подпрограммы & $\mathrm{function}\ F(A_1,A_2,\dots,A_n)\ B$ \cr
				\hline
			\end{tabular}
		\end{center}

		\vspace{1ex}
		\begin{Theorem}[Бём, Якопини]
			Программы, использующие управляющие конструкции последовательности, ветвления и цикла, способны вычислить любую вычислимую функцию.
		\end{Theorem}
	}

	\frame{
		\frametitle{Структурное программирование (продолжение)}

		\textbf{Концепции:}
		\begin{itemize}
			\item
			выделение повторяющегося кода в подпрограммы (процедуры и функции);
			\item
			объединение логически связанных подпрограмм в модули;
			\item
			использование блоков инструкций для контроля области видимости переменных и~функций;
			\item
			проектирование программ сверху вниз;
 			\item
			разделение интерфейсов и имплементаций подпрограмм в модулях.
		\end{itemize}

		\vspace{1ex}
		\textbf{Языки программирования:} Algol, C, Pascal, Ada, FORTRAN, PL/I.
	}

	\subsection{ООП}

	\frame{
		\frametitle{Объектно-ориентированное программирование}

		\begin{Definition}
			\textbf{Объектно-ориентированное программирование} \engterm{object-oriented programming} — парадигма, 
			согласно которой программа представляется в виде взаимодействующих объектов.
		\end{Definition}

		\vspace{1ex}
		\begin{Definition}
			\textbf{Объект} — сильная связь между структурами данных и методами ($\simeq$ функциями), 
			обрабатывающими эти данные. Составляющие объекта: 
			\begin{itemize}
				\item идентификатор; 
				\item свойства;
				\item методы.
			\end{itemize}
		\end{Definition}

		\vspace{1ex}
		\textbf{Особенность:} развитие структурного программирования с объединением в объектах данных и~поведения.
	}

	\frame{
		\frametitle{Концепции ООП}

		\begin{itemize}
			\item 
			\textbf{объекты};
			\item 
			\textbf{инкапсуляция} — скрытие информации от внешних (по~отношению к~системе или~объекту) сущностей;
			\item
			\textbf{наследование} — повторное использование методов работы с~данными в~различных условиях; 
			дополнение функциональности объектов;
			\item
			\textbf{полиморфизм подтипов}\footnote{%
				Полиморфизм есть и в структурном программировании, 
				но \extlink{https://en.wikipedia.org/wiki/Ad_hoc_polymorphism}{другой}.} 
				— возможность использования наследованных от~объекта потомков в~том~же~контексте, что~и~сам~объект;
			\item
			\textbf{динамическая диспетчеризация} — выбор, какую реализацию виртуального метода следует вызывать, 
			во время исполнения программы;
			\item
			\textbf{открытая рекурсия} — переменная \code{this}/\code{self}, позволяющая вызывать методы объекта из~других~методов.
		\end{itemize}
	}

	\frame{
		\frametitle{Виды ООП}

		\begin{itemize}
			\item
			ООП на основе \textbf{классов} — вид ООП, в котором поведение объектов и~наследование 
			определяется с помощью классов (начальный набор данных + поведение).

			\vspace{0.5ex}
			\textbf{Языки программирования:}
			\begin{itemize}
				\item C++;
				\item Java;
				\item C\#;
				\item Python.
			\end{itemize}

			\vspace{0.5ex}
			\item
			ООП на основе \textbf{прототипов} — вид ООП, в котором наследование поведения 
			осуществляется с помощью клонирования существующих объектов (прототипов).

			\vspace{0.5ex}
			\textbf{Языки программирования:}
			\begin{itemize}
				\item JavaScript;
				\item Lua.
			\end{itemize}
		\end{itemize}
	}

	\frame{
		\frametitle{Виды ООП (продолжение)}

		\begin{center}\small
			\begin{tabular}{|*{3}{p{0.275\textwidth}|}}
				\hline
				\centering\textbf{Характеристика} & \multicolumn{2}{c|}{\textbf{Вид ООП}} \cr
				\cline{2-3}
				& \centering\textbf{На основе классов} & \centering\textbf{На основе прототипов} \cr
				\hline
				Наследование & создание подклассов & клонирование прототипа \cr
				\hline
				Содержимое объектов & определяется классом & \raggedright определяется конкретным объектом \cr
				\hline
				\raggedright Связь между данными и~поведением & сильная & слабая \cr
				\hline
				Интерфейсы & подвид классов & duck typing \cr
				\hline
			\end{tabular}
		\end{center}
	}

	\section{Примеры}

	\subsection{Императивное и декларативное программирование}

	\frame{
		\frametitle{Императивное и декларативное программирование}

		\textbf{Вычисление чисел Фибоначчи — императивная программа (Python):}
		\lstinputlisting[language={[doc]python}]{code-fib.py}

		\vspace{1ex}
		\textbf{Вызов:}
		\begin{quote}\upshape
			\code{>>> fib(10) \\ 55}
		\end{quote}
	}

	\frame{
		\frametitle{Императивное и декларативное программирование}

		\textbf{Вычисление чисел Фибоначчи — декларативная программа (Prolog):}
		\lstinputlisting[language=prolog]{code-fib.pro}

		\vspace{1ex}
		\textbf{Вызов:}
		\begin{quote}\upshape
			\code{?- fib(10, X). \\ X = 55.}
		\end{quote}
	}

	\subsection{Процедурное программирование и ООП}

	\frame{
		\frametitle{Процедурное программирование и ООП}

		\begin{figure}
			\begin{tikz*}[%
	class/.style={draw,rectangle split,rectangle split parts=3,align=flush left}
]
	\node(shape) [class,text width=7.5em] {%
		\hfill\textbf{\textit{Shape}}\hfill\strut{}
		\nodepart{two}
		center
		\nodepart{three}
		\textit{draw()} \\
		\textit{fill()} \\
		drawAndFill()
	};

	\node(circle) [class,below left=4em and 1em of shape,text width=7.5em] {%
		\hfill\textbf{Circle}\hfill\strut{}
		\nodepart{two}
		radius
		\nodepart{three}
		draw() \\
		fill()
	};

	\node(rect) [class,below right=4em and 1em of shape,text width=7.5em] {%
		\hfill\textbf{Rectangle}\hfill\strut{}
		\nodepart{two}
		width \\
		height
		\nodepart{three}
		draw() \\
		fill()
	};

	\node (_tmp) at ($ (shape.south) + (0,-2em) $) [coordinate]{};
	\draw (circle.north) |- (_tmp);
	\draw (rect.north) |- (_tmp);
	\draw[->,>=open triangle 60] (_tmp) -- (shape.south);
\end{tikz*}

			\caption{Структурная модель предметной области}
		\end{figure}
	}

	\frame{
		\frametitle{Процедурное программирование и ООП}

		\textbf{Описание ПрО — процедурный стиль (C):}
		\lstinputlisting[language=c]{code-shapes.c}
	}

	\frame{
		\frametitle{Процедурное программирование и ООП}

		\textbf{Описание ПрО — ООП (Java):}
		\lstinputlisting[language=java]{code-shapes.java}
	}

	\frame{
		\frametitle{Процедурное программирование и ООП}

		\textbf{Концепции ООП (Java):}
		\lstinputlisting[language=java]{code-shapes-use.java}
	}

	\subsection{Классы и прототипы}

	\frame{
		\frametitle{Классы и прототипы}

		\textbf{Описание ПрО — прототипы (JavaScript):}
		\lstinputlisting[language=javascript]{code-shapes.js}
	}

	\frame{
		\frametitle{Классы и прототипы}

		\textbf{Особенности прототипов (JavaScript):}
		\lstinputlisting[language=javascript]{code-shapes-use.js}
	}

	\section{Заключение}

	\subsection{Выводы}
	
	\frame{
		\frametitle{Выводы}
		
		\begin{enumerate}
			\item
			Парадигма программирования определяет общий стиль написания программ. 
			По~своей сути парадигмы играют в проектировании и~конструировании ту~же~роль, 
			что~и~модель жизненного цикла в~планировании разработки ПО.

			\vspace{0.5ex}
			\item
			Выделяют две основных группы парадигм программирования: декларативные (определяют цель, но не~метод ее~достижения) 
			и~императивные (определяют и~то, и~другое).

			\vspace{0.5ex}
			\item
			Объектно-ориентированное программирование — основная парадигма в~разработке современных прикладных приложений. 
			Ее особенность — представление предметной области в виде объектов, которые сочетают в себе данные и поведение.
		\end{enumerate}
	}
	
	\subsection{Материалы}
	
	\frame{
		\frametitle{Материалы}
		
		\begin{thebibliography}{9}
			\bibitem[1]{1}
			Abelson, Hal; Sussman, Gerald Jay. 
			\newblock Structure and Interpretation of Computer Programs. 
			\newblock {\footnotesize \url{http://mitpress.mit.edu/sicp/}} 
			\newblock {\footnotesize (функциональное программирование)}

			\bibitem[2]{2}
			Н. Вирт. 
			\newblock Алгоритмы + структуры данных = программы. 
			\newblock {\footnotesize (структурное программирование)}

			\bibitem[3]{3}
			Meyer, Bertrand. 
			\newblock Object-Oriented Software Construction.
			\newblock {\footnotesize (ООП)}
		\end{thebibliography}
	}
	
	\frame{
		\frametitle{}
		
		\begin{center}
			\Huge Спасибо за внимание!
		\end{center}
	}
\end{document}
