\begin{tikz*}[%
	every node/.style={align=center},
	block/.style={rectangle split,draw,rectangle split parts=2,align=center},
	every two node part/.style={font=\small,align=left},
	label/.style={font=\footnotesize}
]
	\node(ctrl) [block] {%
		\textbf{Контроллер}
		\nodepart{two}
		Обработка HTTP-запросов; \\
		системная логика; \\
		проверка данных форм.
	};
	\node(view) [block,right=8em of ctrl] {%
		\textbf{Представление} 
		\nodepart{two}
		Создание веб-страниц \\
		на основе шаблонов; \\
		отображение форм.
	};
	\node(model) [block,below right=8em and 4em of ctrl,anchor=center] {%
		\textbf{Модель}
		\nodepart{two}
		Взаимодействие с СУБД; \\
		логика предметной области.
	};
	
	\draw[->] (ctrl) -- node[label,below]{выбор представления} (view);
	\draw[->] (ctrl.south) |- node[label,pos=0.3,left,align=right]{запросы на изменение/ \\ создание/удаление} (model.west);
	\draw[->] (model.east) -| node[label,pos=0.7,right,align=left]{уведомление \\ об изменениях} (view.south);
	\node(_tmp) at ($ (model.north east) + (-1em, 0) $) [coordinate] {};
	\draw[<-] (_tmp) -- node[label,left,align=right] {запрос состояния} (_tmp |- view.south);
	
	\node(browser) [ellipse,draw,above right=3em and 4em of ctrl,anchor=center] {Веб-браузер};
	\draw[->] (view) to (browser);
	\draw[->] (browser) -- node[label,right] {польз. события} (ctrl);
\end{tikz*}
